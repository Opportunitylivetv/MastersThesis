          \chapter{Results \& Future Work}\label{results}


\section{Online Demo}

Our implementation of this paper and a brief solution demonstration is available online at the following url (Mozilla Firefox, Google Chrome, or Apple Safari recommended for full compatibility):

\vspace{0.2in}

\begin{centering}
\url{http://petercottle.com/liquidGraph/index.html?demo}
\end{centering}

\vspace{0.2in}

Another solution demonstration for a more complex part is available at this url:

\vspace{0.2in}

\begin{centering}
\url{http://petercottle.com/liquidGraph/yc.html?demo}
\end{centering}

\vspace{0.2in}


\newpage


\myfigure{width=0.85\linewidth}{demoFoundSolution}{A screen capture from our implementation, available online.}

Our goal with the implementation of this work was to create an interactive, dynamic experience that visualized both solution search and final results. Creating this experience required a significant investment in graphical design and subroutines solely dedicated to visualization; it also required that our implemenation be designed to run asynchronously between frame draws.

The result of this design objective was a highly visual experience for the user at the expense of raw performance. Consequently, we remind readers that the following run time analysis was conducted on an implementation optimized for visuals rather than performance.

  \subsection{Run Time}

Because our algorithm performs a significant amount of graphics processing, absolute time benchmarks for our algorithm do not provide as much utility for analysis. Consequently, we supplement our analysis with percentage breakdowns of CPU time between high-level functions to give readers an idea of which subroutines require the most computation.

The following CPU profile was collected during a solution search on a workpiece with 180 total vertices, 55 concave vertices, and 21 polygons. The solution was obtained in 14.64 seconds on a Macbook Air with a 1.86 GHz Intel Core 2 Duo and 2 GB of 1067 MHz DDR3 RAM. This benchmark time, however, was obtained while a CPU profiler was running (in addition to other applications) and is not representative of the lower-bound runtime for the algorithm.

In Table \ref{resultsTable}, CPU ``Total Time'' is defined as the sum of computational time spent inside a given function call and all other functions called from that location. This means that high-level functions that simply call other parts of the program will have a large total time, despite very little computational time being spent inside them.

\begin{table}[H]\label{resultsTable}
\centering
\def\arraystretch{1.2}
\begin{tabular}{|l|c|c|}
\hline
High Level Category & CPU Total Time & Function \\ \hline
Idle Time & 53.92\% & (time spent idle waiting for next animation frame) \\
Program & 46.08\% & (total time spent processing in Javascript) \\
Graphics & 21.35\% &  (total time spent in graphics processing) \\ \hline
\textbf{Physical Simulation} & ~ &  ~ \\ \hline
~ &  2.8\% &  Edge-Parabola intersection \\
~ &  0.64\% & Quadratic equation solver \\
~ &  0.30\% &  Edge-Point containment test \\
~ &  0.38\% & Particle sliding equation \\
~ &  0.11\% & Particle collision \\ \hline
\textbf{A.I. Search} & ~ & ~ \\ \hline
~ &  3.83\% & Gravity turn sampling \\
\hline
\end{tabular}
\caption{CPU Time breakdown for a solution search}
\end{table}

  \subsection{Ranking Heuristic Performance Improvement}

Our plan ranking heuristic presented in Chapter \ref{multipleExtension} produced great performance improvements during multiple-particle search. Here we present a table summarizing the running time difference between our implementation with and without this plan heuristic.

\begin{table}[H]\label{rankingHeuristicResults}
\centering
\def\arraystretch{1.5}
\begin{tabular}{|c|c|c|c|c|}
\hline
Test Piece URL & \specialcell{ Number \\ of Particles} & \specialcell{Time without \\ Ranking Heuristic} & \specialcell{Time with \\ Ranking Heuristic} & Factor Improvement \\ \hline
$index.html?demo$ & 4 & 20.672s & 7.009s & 2.949 \\ \hline
$yc.html?demo$ & 2 & 100.947s & 62.104s & 1.625 \\ \hline
\end{tabular}
\end{table}

  \subsection{Constants}

Our implementation uses the following values for simulation constants:

\begin{table}[H]\label{constantsTable}
\centering
\def\arraystretch{1.1}
\begin{tabular}{|c|c|c|}
\hline
Constant Description & Symbol & Value \\ \hline
Particle Elasticity & $\kappa$ & 0.5 \\
Perpendicular Edge Velocity & $\epsilon$ & 0.9\\ \hline
\end{tabular}
\end{table}


  \subsection{Scenario \#4 Rejection}

Our preliminary tests indicate that our restriction on scenario \#4 during simulation does not have a large effect on obtaining a solution. While we are not able to determine if workpieces exist in which every solution contains a scenario \#4 particle path, we present a summary of the number of samples rejected due to this restriction below in Table \ref{scenario4rejects}.

\begin{table}[H]\label{scenario4rejects}
\centering
\def\arraystretch{1.2}
\begin{tabular}{|c|c|c|c|}
\hline
Test Piece URL & Total Samples & Scenario \#4 Samples Rejected & Percentage \\ \hline
$index.html?demo$ & 3400 & 854 & 25.12\%\\
$yc.html?demo$ & 1200 & 264 & 22\% \\ \hline
\end{tabular}
\end{table}

It is also important to note that manufacturers may prioritize obtaining \emph{any} solution to the draining problem in a reasonable amount of time over obtaining \emph{the} optimal solution (at the expense of time). If this is the case, it is reasonable to reject scenario \#4 for performance reasons.

\section{Future Work \& Discussion}

Further optimizations and improvements could be made to this work, namely to enhance the usability of the results and increase efficiency.

  \subsection{Heuristics for A* Search}

One standard improvement to uniform cost search is the addition of a consistent, admissible heuristic to convert the search into A* search. A heuristic is simply an estimation of the remaining cost from a state to the goal state. Admissibility of a heuristic requires that the heuristic never over-estimates the remaining cost, and consistency requires that the heuristic does not rapidly change values between states.

Heuristics are, by nature, very dependent on the problem they are being applied to. Many heuristics exist for positional search problems and other common search formulations; since our search formulation is rather unique, there is not as much prior work to draw from \cite{heuristics}.

One initial idea for a heuristic would be to use the free-fall time from the current concave vertex to the outside of the workpiece bounding box. This essentially would be the optimistic estimation that the particle would free-fall immediately from the concave vertex at the next turn. While this heuristic would be admissible and consistent, it does not take into account workpiece geometry; consequently, the information. It is probable that better heuristics exist for this search formulation.

  \subsection{Bounding Box Method Adaption for Particle Path Intersections}

When intersecting rays with geometric primitives, most implementations using bounding-box methods to reduce the number of primitives tested against the ray. This bounding-box idea could be adapted to parabolic paths as well; determining which bounding boxes are needed still presents a fairly reasonable computational load, but great gains in efficiency could be made for complex input with many edges.

  \subsection{Adaption to 3D - Fixed Axis}

Adapting this work to 3D polygonal meshes when the rotation axis of the workpiece is fixed is mainly an exercise in implementation. The primary difference is that parabolic paths would then need to be intersected with planes in 3D space rather than edges in 2D space.

The rest of the approach, from solution search to particle simulation, would remain the same.

  \subsection{Adaption to 3D - Free Axis}

Adapting this work to 3D polygonal meshes where the workpiece can rotate about \emph{any} rotation axis at any time presents a substantially more difficult problem. The chief obstacle here is that there is an explosion of possible turns available from any given concave vertex. The final gravity vector of the turn, $g_{end}$, could be sampled from an entire sphere rather than two small arcs in a planar circle.

There may be ways to intelligently reduce the sample space and still maintain a representative coverage of the kinetic paths attainable from a concave vertex, but much work is needed before those methods can be validated.

