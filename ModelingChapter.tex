						\chapter{Modeling Approach}

Solving either objective of the drainability problem requires that the behavior of fluid as it travels through the workpiece be modeled in some way. This behavior is complex; trapped gases, viscous effects, and fluid mechanics all interact during workpiece draining. There are many ways to model this fluid behavior, and each choice comes with both advantages and disadvantages.

Advanced fluid simulation methods like computational fluid dynamics or smoothed-particle hydrodynamics approximate fluid behavior with high accuracy, but their computation is quite expensive in both time and memory. This is primarily because both are integration methods -- their accuracy relies on integrating many times over a small timestep.

Drainability analysis requires many different rounds of fluid simulation in many different configurations; consequently, it is worthwhile to sacrifice accuracy in the fluid model to obtain the ability to simulate many times during algorithm execution. Yasui et al.'s existing work chooses this exact tradeoff and produces meaningful and worthwhile results \cite{Yasui2011}. We choose a similar approach of a simplified fluid model that is used in many different simulations.

\section{Straight-Line Multiple-Particle Model}

	\subsection{Core Assumptions}

Yasui et al. \cite{Yasui2011} chose a simplified particle modeling approach with the following core assumptions:

\begin{itemize}
	\item each fluid body within the workpiece is approximated as a single particle,
	\item water particles only travel in straight lines and do not maintain a kinetic state,
	\item the workpiece rotates infinitesimally slow in one direction for an infinite amount of time.
\end{itemize}

	\subsection{Modeling Choice Advantages and Disadvantages}

This modeling approach allowed Yasui et al. to reduce the fluid draining problem down to a particle draining problem. Furthermore, particle simulation within the workpiece is greatly simplified; particles always travel in straight lines and do not accumulate kinetic energy. This means that particles are always traveling either parallel to gravity or along a workpiece edge.

Consequently, particle simulation in this choice of model was essentially reduced to finding intersections between lines and surfaces. An intersection test between a line and a surface runs in constant time; consequently, intersecting a line with the entire workpiece is linear in the number of triangles. Yasui et al. used bounding-box methods to reduce this complexity to sub-linear in the number of triangles, meaning each intersection test (and particle simulation as a whole) was then computationally inexpensive \cite{Yasui2011}.

Optimizing particle simulation gave Yasui et al. the ability to perform many particle simulations during analysis; the end result is that for every possible rotation axis, a particle is simulated out of every concave vertex in the workpiece \cite{Yasui2011}.

This thoroughness of simulation produces a set of results that describes the workpiece drainability of any rotation axis. This analysis serves great utility when designing for manufacturability; engineers can quickly determine if a geometry change affects drainability of the workpiece before the workpiece is even manufactured.

Yasui et al.'s model assumes infinitesimally slow rotation of the workpiece for an infinite amount of time. Because of the assumption about the workpiece rotation, Yasui et al.'s results produce a binary value of drainability about a particular axis rather than a direct control sequence.

If this assumption about the rotation of the workpiece could be relaxed and a  control sequence of how to drain the workpiece could be produced, manufacturers could further optimize the draining stage of product manufacturing.

This work aims to produce those complementary results while simultaneously improving the kinetic model of fluid behavior.

\section{Parabolic Single-Particle Model}

Our work utilizes a different (but related) model of the workpiece draining problem. The results produced are complementary to Yasui et al.'s results and are ideally used in conjunction.

This model is based on the following assumptions:

\begin{itemize}
	\item each fluid body within the workpiece is approximated as a single particle,
	\item water particles travel in parabolic paths that are influenced by acceleration,
	\item the workpiece can be rotated in either direction along its rotation axis,
	\item the workpiece can alternate between rotating for any amount of time or holding a specific orientation for any amount of time,
	\item the workpiece is a two dimensional polygon.
\end{itemize}

These simplifying assumptions define the model that we use. Additionally, one more restriction is imposed due to simulation constraints:

\begin{itemize}
	\item particles may not travel through open space during workpiece rotation.
\end{itemize}

The reasons for this additional assumption are detailed in Chapter \ref{physicalChapter}.

In the following chapters, we first present our methods assuming a single particle will be drained for simplicity of presentation. We then extend this work to multiple particles in Chapter \ref{multipleExtension}.

	\subsection{Modeling Choice Advantages and Disadvantages}

This modeling choice requires a different approach to drainability analysis than the approach used by Yasui et al.; our approach, however, draws inspiration from their previous work. Our different modeling choice allows us to produce a specific sequence of rotations to drain the workpiece (solving the second objective); the existence of this sequence of rotations then answers the drainability query (the first objective).

The key components of our approach break down into the water particle simulation (presented in Chapter 3) and the search for a drainability sequence (presented in Chapter 4).

\section{Algorithm Inputs and Outputs}

Before proceeding, it is important to formally define the inputs to the draining problem and the outputs that our work produces.

The draining problem is defined as removing bodies of fluid from a given workpiece. Consequently, our work accepts the following variables as input:

\begin{itemize}
\item a collection of 2D polygons that represent the workpiece,
\item a set of concave vertices representing the bodies of water to drain.
\end{itemize}

Our work also accepts the desired rotation speed of the workpiece as an input. While this value could theoretically vary in time, we decide to use an algorithm-wide constant for simplicity.

Our work then produces the following two results:

\begin{itemize}
\item whether or not a solution exists (answering the ``drainability'' query),
\item a $C0$ continuous function of rotation angle through time. If the workpiece is rotated according to this function, the input particles will be drained; consequently, this function is referred to as the control sequence.
\end{itemize}

Figure \ref{controlsquence} illustrates the control sequence that is generated from our work.

\myfigure{width=0.8\linewidth}{controlsquence}{The $C0$ continuous function of rotation angle through time that will drain the input particles. This control sequence is the solution to one of the example geometries provided in the results chapter (Chapter \ref{results})}

The following chapters present how these results are obtained.

