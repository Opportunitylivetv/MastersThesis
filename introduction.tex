							\chapter{Introduction \& Motivation}

A paragraph about manufacturing work pieces and jet cleaning
\\

A paragraph about draining the fluid after cleaning. Oven approach vs rotating / draining approach.
\\

Manufactureres and designers have two key objectives to accomplish when presented with this manufacturing process. The first is to analyze the ``drainability'' of the workpiece about a given axis. Drainability is defined as the ability for the workpiece (or part) to be drained by rotation, usually in reference to a given rotation axis. Existing work has made great progress in this domain; Yasui et al. \cite{plot}'s work is capable of producing a map of which rotation axes will fully drain the workpiece under an infinite number of rotations. These results can be quite useful when designing rotation fixtures or altering the geometry of a workpiece, despite the algorithm using a greatly simplified model of true fluid behavior.

While the drainability analysis for a workpiece is useful, it does not produce any bounds on the time required to drain the workpiece. Likewise, it also does not specify a rotation speed or total rotation angle. For this reason, manufacturers would also like to accomplish the second objective of the drainability problem -- achieving a specific sequence of rotations that fully drains the workpiece under a given rotation axis.

This sequence of rotations would be immediately useful on the manufacturing floor to input as a control sequence to a workpiece rotator. Little progress has been made in this domain and is the focus of our work presented in this paper. If both of these objectives can be accomplished, the combined solution would serve great utility in reducing time and energy required to manufacture a given part.

