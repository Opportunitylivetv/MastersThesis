							\chapter{Introduction \& Motivation}

A paragraph about manufacturing work pieces and jet cleaning
\\

A paragraph about draining the fluid after cleaning. Oven approach vs rotating / draining approach.
\\

The key obstacle for manufacturers is both to analyze the ``drainability'' of a workpiece and produce a series of actions that will fully drain the workpiece.

Drainability in this sense refers to the ability for a part to be fully drained by rotations about a particular axis. Existing work has made great progress in this domain; Yasui et all \cite{plot}'s work is capable of producing a map of which rotation axes will fully drain the workpiece under an infinite number of rotations. These results can be quite useful when designing rotation fixtures or altering the geometry of a workpiece, despite the model being a simplified version of the true fluid behavior.

More progress is needed however in the domain of producing a series of actions that would drain the part. This work aims to produce those useful results by taking a different approach to existing work. (TODO: more here)
