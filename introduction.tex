							\chapter{Introduction \& Motivation}


During the manufacturing process for engine components, many byproducts are produced that contaminate the final workpiece \cite{Hancock94}. These byproducts can include sand from sand casting, chips from additional machining, and burrs from hole-cutting \cite{Arbelaez}.
These byproducts are commonly cleaned off with high pressure water-jets; while these jets are effective at removing contaminants, they introduce large volumes of water that may settle in concave regions of the workpiece \cite{Arbelaez}. This fluid must be removed before the final stages of the manufacturing process can continue; the removal of this fluid is a non-trivial task due to complex workpiece geometry \cite{Avila} \cite{Yasui2011}. Removing this fluid from the workpiece is referred to as the ``workpiece drainability'' problem in our field.

Manufacturers and designers have two key objectives to accomplish when presented with the drainability problem. The first is to analyze the ``drainability'' of the workpiece about a given axis. Drainability is defined as the ability for the workpiece to be drained by rotation, usually in reference to a given rotation axis. Existing work has made great progress in this domain; Yasui et al.'s work is capable of producing a map of which rotation axes will fully drain the workpiece under an infinite number of rotations \cite{Yasui2011}. These results can be quite useful when designing rotation fixtures or altering the geometry of a workpiece, despite the algorithm using a simplified model of true fluid behavior.

While the drainability analysis for a workpiece is useful, it does not produce any bounds on the time required to drain the workpiece. Likewise, it also does not specify a rotation speed or total rotation angle. For this reason, manufacturers would also like to accomplish the second objective of the drainability problem -- achieving a specific sequence of rotations that fully drains the workpiece under a given rotation axis.

This sequence of rotations would be immediately useful on the manufacturing floor to input as a control sequence to a workpiece rotator. Existing work has mainly focused on analyzing drainability \cite{Yasui2011} or the number of machining operations needed to attain drainability \cite{Aloupis_draininga}. Consequently, the focus of our work presented in this paper is to produce a specific sequence of rotations for drainability. If both halves of the drainability problem can be solved, the combined solution would serve great utility in reducing time and energy required to manufacture a given part.

