						\chapter{Physical Simulation of Water Particles}

Water particle simulation is the underlying core of drainability analysis -- the behavior of the fluid (or water particles in this case) must be simulated in order to analyze when and where the fluid can leave the workpiece.

Here we present the popular approach to simulating straight-line water particles based on parametric rays (the approach used in Yusuke et all's work) and then our modification of this approach to fit parabolic water particles.

\section{Reduced Ray Approach}

	\subsection{Underlying Kinetic Model}

Individual water particles under an acceleration field $a$ with initial position $x_0$ and initial velocity $v_0$ can be modeled with a simple kinetic equation.

\myequation{
	x(t) = x_0 + v_0 \cdot t + \frac{1}{2} a \cdot t^2
}{
	\label{eq:basickinematic}
}

Equation \eqref{eq:basickinematic} shows the basic kinematic model of a water particle.


Note that this model omits the effect of aerodynamic drag ($-\rho v^2 \cdot t$) on the particle. The velocities achieved in workpiece draining produce fairly negligible aerodynamic effects, this omission greatly simplifies calculating the trajectory of a particle.

	\subsection{Reduction to Rays}

In order to obtain particles that always travel in straight lines, many researchers chose to omit the acceleration term from \eqref{eq:basickinematic}. Under this condition, these particles can effectively be modeled as ``rays.''

\myequation{
x(t) = x_0 + v_0 \cdot t	
}{
	\label{eq:kinematicDropAccel}
}

Equation \eqref{eq:kinematicDropAccel} shows this simplified ray model. Once the motion of a particle under un-obstructed movement can be easily produced, the primary challenge of particle simulation is finding the collision points of a particle's path.

	\subsection{Parametric Equations (rays)}

There are many ways to find these intersection (or ``collision'') points. One common approach is to model the ray equation as a parametric equation.

% TODO -- cite a common text here!

\myequation{
	\vec{x}(t) = \vec{x}_0 + \vec{v}_0 \cdot t
} {
	\label{eq:parametricRay}	
}

Equation \eqref{eq:parametricRay} shows this modeling choice with vectorized positions and a single parameter $t$ that describes the motion of the particle through space.

\myfigure{rayexample}{A visual depicition of a ray's component vectors and its path through time}

	\subsection{Ray Tracing}

Once particles are modeled as parametric rays, all the existing techniques and libraries from ``ray-tracing'' (a standard approach to producing 3D computer graphics) can be used to find intersection points.

Ray-tracing produces 3D computer graphics by sending out rays from a camera location. If one of these rays intersects a geometric primitive in the scene, the resulting color of that ray is calculated and stored in a pixel table. These ray-primitive intersections are the fundamental computational bottleneck for ray tracing; consequently, much work has been done to select the optimal intersection test approach. Parametric modeling of geometric primitives combined with parametric rays are the industry standard.

	\subsection{Geometric Primitive Intersections}

Because these rays are defined parametrically, they can be substituted into the parametric definition of a geometric primitive to obtain exact solutions of intersection points. This is the primary advantage of parametric equations over integration schemes -- intersections between a ray and a geometric primitive can be solved for directly in constant time. 

For example, the parametric definition of a sphere with center $C$ and radius $r$ is given in equation \eqref{eq:sphereParametric}.

\myequation{
	|\vec{X} - \vec{C}|^2 - r^2 = 0
} {
	\label{eq:sphereParametric}
}

In equation \eqref{eq:sphereParametric}, all points $\vec{X}$ that satisfy the equation define the surface of the sphere. In order to solve for the intersection of a ray and a sphere, the parametric equation of the ray is substituted into \eqref{eq:sphereParametric} for $X$. This produces \eqref{eq:sphereParametricSubbed}.

\myequation{
	|(\vec{x}_0 + \vec{v}_0 \cdot t) - \vec{C}|^2 - r^2 = 0
} {
	\label{eq:sphereParametricSubbed}
}

\myfigure{raysphere}{A demonstration of a ray intersecting with a sphere.}

Equation \eqref{eq:sphereParametricSubbed} can be expanded algebraically to yield an equation that is quadratic in $t$. This resulting equation can then be solved using the standard quadratic formula:

\myequation{
	\frac{-b \pm \sqrt{b^2 - 4ac}}{2a} = t
} {
	\label{eq:quadraticFormula} 
}

This constant-time solution for the intersection between a parametric equation and geometric primitive will be crucial for later parts of this paper.

\section{Kinetic Parametric Approach}

While this straight-line particle approach is convenient, our model aims to examine particles with more realisitc kinetic properties and paths. Consequently, we choose to not omit the acceleration term from the kinetic model of a particle. This means we combine a parametric approach with \eqref{eq:basickinematic} to produce \eqref{eq:freeFall}.

\myequation{
	\vec{x}(t) = \vec{x}_0 + \vec{v}_0 \cdot t + \frac{1}{2} \vec{a} \cdot t^2
}{
	\label{eq:freeFall}
}

\section{Equation Types}

During the course of simulation, a particle may be traveling in a variety of conditions. The two primary conditions that determine the type of equation used are:

\begin{itemize}
	\item Whether the workpiece is rotating
	\item Whether the particle is traveling along a workpiece surface or through open space.
\end{itemize}

These two conditions, when combined in all possible ways, produce four scenarios of particle simulation that must be modeled. The following sections cover each scenario and detail the equation used.

		\subsection{Free Fall Equation}

When the particle is moving through open space with no workpiece rotation, we consider that particle to be in basic ``free-fall.'' Equation \eqref{eq:freeFall} gives us the equation that models this scenario; the particle travels through empty space with a constant acceleration field. The resulting motion is a parabola.

\myfigure{freefallequation}{Example path traced out by the ``free-fall'' equation}

		\subsection{Sliding Equation}

During simulation, a particle may also begin to travel along an edge if it has a perpendicular velocity less than $\epsilon$. This produces another scenario of particle simulation -- a particle traveling along an edge when the workpiece is not rotating.


In this scenario, the acceleration of the particle in free-fall is projected along the surface of the workpiece. If this projected acceleration is zero, the particle will stay in place; if not, the particle will begin to ``slide'' along this edge.

We achieve the modeling of this scenario by projecting the acceleration in \eqref{eq:freeFall} along the workpiece surface, which produces a straight-line path.

Note that while the particle is now traveling along an edge, it is essentially the same as the freefall equation with a new projected acceleration.

\myequation{
	\vec{x}(t) = \vec{x}_0 + \vec{v}_{0} \cdot t + \frac{1}{2}\vec{a}_{projected} \cdot t^2
}{
	\label{eq:sliding}
}

\myfigure{slidingequation}{Demonstration of the sliding equation.}

With these two formulations, the water particle's path can be simulated on both edges and in open space when the acceleration field is constant (e.g. the workpiece is not rotating).

		\subsection{Rotation}

Here we begin to examine the scenarios where the workpiece is rotating during particle simulation.

In this paper, we choose our frame of reference to be the $X$ $Y$ axes that define the workpiece geometry. This means that when the workpiece rotates, our frame of reference stays fixed to the workpiece. Consequently, rotation changes the orientation of the acceleration field changes in time. 

We see now how this rotating acceleration field affects the following scenarios.

		\subsection{Concurrent Rotation \& Sliding Equation}

In this scenario, the particle is traveling along a workpiece edge and the workpiece is rotating.

We know that the acceleration vector is projected along the edge during edge travel. This means the direction of the acceleration in this scenario is fixed; only the magnitude of the projected acceleration varies as the workpiece rotates. Consequently, rotation of the acceleration field only results in the magnitude of the projected acceleration changing.

This leads to a special case of particle motion, referred to as ``concurrent rotation and sliding.'' Equation \eqref{eq:slidingRotating} shows the resulting formulation of this scenario.

\myequation{
	\vec{x}(t) = \vec{x}_0 + \vec{v}_{0} \cdot t + \frac{1}{2}\hat{a}_{projected}  \cdot t^2 \cdot |a_{projected}|(t)
} {
	\label{eq:slidingRotating}
}

In equation \eqref{eq:slidingRotating}, $\hat{a}_{projected}$ is a constant unit-vector in the direction of the edge. The $|a_{projected}|(t)$ term scales as the acceleration field changes direction. If the rotation occurs at a constant rate, the magnitude term can be modeled as
$$
mag_{accel} = cos(\omega \cdot t)
$$

where $omega$ is the angular velocity.

		\subsubsection{Concurrent Rotation \& Sliding Equation Intersection}

In equation \eqref{eq:slidingRotating}, we no longer have a formulation of particle motion that is strictly quadratic in $t$. With the additional $cos$ term, this makes it quite difficult to solve for particle-primitive intersections. We can, however, easily subtitute in values of $t$ to determine the position of the particle at a given time.

The ability to determine where the particle is at a given time will be crucial for our methods later; this scenario will still be simulated, despite the inability to intersect these trajectories with geometric primitives.

		\subsection{Concurrent Rotation \& Free-Fall Equation}

The fourth scenario is when the particle is traveling through open space and the workpiece is rotating in time. In this case, the acceleration vector is no longer projected along an edge as shown in \eqref{eq:slidingRotating}. Thus, the acceleration vector has a direction that changes in time (with a fixed magntiude), as shown in \eqref{eq:freefallRotating}.

\myequation{
	\vec{x}(t) = \vec{x}_0 + \vec{v}_{0} \cdot t + \frac{1}{2}\vec{a}(t)  \cdot t^2
} {
	\label{eq:freefallRotating}
}

Here, the $\vec{a}(t)$ term describes the acceleration vector over time as the workpiece rotates. Assuming a constant rotation, this term breaks down to

$$
a_x = cos(\alpha t) \cdot |a_g|
$$

$$
a_y = sin(\alpha t) \cdot |a_g|
$$

		\subsubsection{Assumption \#1 - No concurrent Rotation + Freefall}

These non-linear terms produce a much more complicated equation; again, we can no longer combine this equation with geometric primitives easily to determine  intersection points.

\myfigure{norotationfreefall}{Demonstration of the particle path during concurrent rotating and free-fall}



TODO -- now we can't simulate this because the particle is in free fall and will collide with some geometric primitive, which we can't solve for performantly. In sliding + rotating it doesn't matter, because the particle is already on an edge and not going to collide with anything (besides the adjacent edge, which is easy to lookup).

	\subsection{Elastic Collisions}

	Collisions are now elastic, meaning particles maintain a perpendicular velocity when colliding with an edge.

		\subsubsection{Planar Collision}

		When colliding simply on an edge, they bounce and transition back into freefall.

\myfigure{planarcollision}{Particle Edge Planar Collision.}

		\subsubsection{Planar To Sliding Transition}

		When the particles have some $\epsilon$ perpendicular velocity, they transition from freefall to sliding along an edge.

\myfigure{planartoslidingtransition}{Particle Path transition from Free-Fall to Sliding.}

		\subsubsection{Sliding-Edge Collision}

		When sliding along an edge, the particle may encounter another edge. If this edge has a dot product greater than zero, it collides with the edge and enters freefall again.

\myfigure{slidingedgecollision}{Sliding particle collision with adjacent edge at obtuse.}

		\subsubsection{Sliding-Corner Collision}

		If the next edge has a dot product less than or equal to 0, the particle is effectively ``trapped'' as long as the gravity vector points within the edge.

\myfigure{slidingcornercollision}{Sliding particle collision with adjacent edge at acute angle.}

	\subsection{Conservation of Momentum}

		The only source of energy in this demo is the potential energy from gravity. Note that as the workpiece rotates, the gravity vector (and corresponding field) changes. This means that a particle in a low-energy configuration can transition to a high-energy configuration through rotation, meaning that rotation adds energy into the system.

		\subsubsection{Settling Guarantee}

		The only addition of energy is from rotation, so in the absence of rotation, no energy is added to the system. Because the elasiticity $\kappa$ of the particle-edge collision is less than 1, energy dissipates as the particles are simulated throughout the workpiece.

		\subsubsection{Duration of Simulation}

		Because of this energy dissipation, the simulation is guaranteed to terminate in one of two ways.

		\subsubsection{Simulation End -- Concave Vertex}

		Either the particle settles into a concave vertex with a kinetic energy less than $\epsilon$, or

		\subsubsection{Simulation End -- Workpiece Exit}

		The particle exits the workpiece, which is the goal of the simulation.

\section{Results}

	\subsection{Run Time}

	It was fast, here are some numbers

	\subsection{Accuracy Comparison}

	It was accurate in finding intersections that some euler integration schemes would not.

		\subsubsection{With Euler Integration}

\section{Future Work \& Discussion}

	\subsection{Bounding Box Method Adaption}

	You could modify normal bounding box methods to use parametric equations instead.

	\subsection{Bounded Simulation Between Limits}

	There is a possibility that you could simulate between two arbitrary limits and ``sweep'' across the range of kinetic possiblities. Would eliminate the sampling we will see next chapter.
