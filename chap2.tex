								\chapter{Solution Search}

\section{General A.I. Search}

General A.I. Search essentially expands from a start state and proceeds towards a goal.

	\subsection{State Space Formulation}

The state space is the tuple of information that directly encodes the time-variant properties that describe your progress towards the goal. For pacman, the example is his position and the food.
\\

Other parts of the problem that are not time-variant are considered part of the environment (for instance, the walls of pacman or the polygons in our workpiece here).


	\subsection{State Space Size}
The size of the state space is all possible valid combinations of the components of the state space. E.G. for pacman, it's $MN$ for an M x N sized board.

	\subsection{State Space Exploration}

Exploration of a state space happens through a transition function that generates successors from a state from a list of available actions.
\\

For pacman, this means that North/South/East/West produce different states. Transition functions can fail to return a state from an action if that action is invalid.

\section{Adaption of A.I. Search}

Our A.I. Search algorithms will be quite similar, but our state space approach is quite different.

	\subsection{Traditional Formulation}

The traditional formulation for kinematic problems is to define the state of a particle as it's position, velocity, and acceleration. Successor function integrates in time.

	\subsection{Our State Space Formulation}

Our state space formulation is instead either:

\begin{enumerate}
\item The concave vertex the particle is resting in
\item The particle is out of the workpiece
\end{enumerate}

Our state space is thus dramatically reduced in complexity compared to the traditional formulation, meaning our state space size will be reduced dramatically as well.

	\subsection{Exploration}

	Instead, the bulk of the algorithm's work will be in the transition function, allowing us to determine what concave vertices we can reach from a given state.

\section{Transition Function}


	\subsection{Definition}

Transition function returns states.

	\subsection{Sampling}

We will sample from our available gravity directions and simulate a kinetic path from there.

%figureneeded

	\subsection{Representative Coverage Between Limits}

Try to get an even coverage over the space, sample with $\sqrt[3]{x}$

%figureneeded

	\subsection{Graph Search}

Reduce possibilities, prioritize by our ``cost'' for the algorithm.

		\subsubsection{Cost Sensitive Closed List}

Essentially similar to CSCL because it is order-independent.

\section{Search}

Now we can search from a start state outwards

	\subsection{Uniform Cost Search}

Using uniform cost, we always maintain optimality based on our cost.

	\subsection{Cost Function}

Our cost function can be a variety of things based on the backwards path.

		\subsubsection{Time}

Most popular is time, because time is money.

		\subsubsection{Energy - Rotation Angle}

Second most popular is energy based on the rotation angle. This would be appropriate if the workpiece was well balanced but fixture generated a lot of friction while rotating.

		\subsubsection{Energy - Workpiece Center of Gravity}

Another option is based on center of gravity if the energy from the workpiece rotations outweighed the friction in the motors.

	\subsection{Solution}

		Defined as the sequence of gravity transitions from each concave vertex to produce a particle path that exits the workpiece.

\section{Control Sequence Generator}

Now that we have a solution, we need to generate a control sequence for the workpiece rotator.

	\subsection{Sample-defined Rotations}

These rotations are defined for us by the transition function.

	\subsection{Intermediary Rotations}

Between these samples we don't have anything. What we do is just interpolate between the last sample and the beginning of the next sample with a cubic bezier curve.

\section{Results}

	\subsection{Run Time}

Runs fast.

	\subsection{Part Complexity}

Transition function is constant in best case, linear in worst case based on the number of polygon edges.

	\subsection{Demonstration of Solution}

Go to this web address!

\section{Future Work \& Discussion}

Could be improvements

	\subsection{Heuristics for A* Search}

A* search can be added with heuristics. Distance to workpiece bounding box might be a good one, but that's essentially a radially symmetric heuristic, not very informative.
