						\chapter{Physical Simulation of Water Particles}

\section{Basic Formulation}

Here we talk about how water particles must be simulated as part of the drainability analysis calculation.
\\

Many methods -- smoothed hydrodynamic particles, etc etc
\\

Another approach is to simply construct the kinematic equations and integrate in time. Euler's method, etc
\\

We will examine a parametric approach.
\\

	\subsection{Parametric Equations (rays)}

Parametric rays look like this (equation).


	\subsection{Geometric Primitive Intersections}

Once parametric rays are defined, you can easily intersect them with geometric primitives
\myfigure{rayexample}{2.5in}

Example of sphere ray intersection equation.
\myfigure{raysphere}{2.5in}


\section{Previous Work}

Previous work (Yusuke's work) involved a few simplifying assumptions about water particle simulation.
\\

	\subsection{Infinitesimally Slow Rotations}
The first was that rotations would be infinitesimally slow, meaning that gravity direction was always essentially perpendicular to the leading edge of the rotation when a particle fell.
\myfigure{prevworkgravityperp}{2.5in}


	\subsection{Inelastic Collisions}
The second was that particle collisions would be inelastic, meaning that velocities were instantaneously projected onto the plane or edge that they collided with.

\myfigure{prevworkinelasticcollision}{2.5in}


	\subsection{Kinetic Energy Limitation}

The last was that particles never accumulated kinetic energy above an epsilon value, meaning that they did not leave leading edges with a finite velocity. This means that all paths traced out by the particles were straight lines. This allowed for fast particle simulation but unrealistic particle behavior.

\section{Adaption to Finite Velocities}

This paper adapts the particle simulation to finite velocities and rotation speeds while maintaining the performant nature of the simulation.

	\subsection{Parametric Equation Modification}

Now our parametric equation includes an acceleration term:

$$
\vec{x}(t) = \vec{x_0} + \vec{v}t + \vec{a}t
$$

		\subsubsection{Free Fall Equation}
In free-fall, this leaves us with a parabolic equation of the particle's path. Kinematically valid, except for finite


		\subsubsection{Rolling Equation}

		\subsubsection{Concurrent Rotation \& Rolling Equation}

		\subsubsection{Assumption \#1 - No concurrent Rotation + Freefall}

	\subsection{Elastic Collisions}

		\subsubsection{Planar Collision}

		\subsubsection{Rolling-Edge Collision}

		\subsubsection{Rolling-Corner Collision}

	\subsection{Conservation of Momentum}

		\subsubsection{Settling Guarantee}

		\subsubsection{Duration of Simulation}

		%\subsubsubsection{Concave Vertex}

		%\subsubsubsection{Workpiece Exit}

\section{Results}

	\subsection{Run Time}

	\subsection{Accuracy Comparison}

		\subsubsection{With Euler Integration}

\section{Future Work \& Discussion}

	\subsection{Bounding Box Method Adaption}

	\subsection{Bounded Simulation Between Limits}


