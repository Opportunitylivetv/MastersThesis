						\chapter{Physical Simulation of Water Particles}

\section{Basic Formulation}

Here we talk about how water particles must be simulated as part of the drainability analysis calculation.
\\

Many methods -- smoothed hydrodynamic particles, etc etc
\\

Another approach is to simply construct the kinematic equations and integrate in time. Euler's method, etc
\\

We will examine a parametric approach.
\\

	\subsection{Parametric Equations (rays)}

Parametric rays look like this (equation).


	\subsection{Geometric Primitive Intersections}

Once parametric rays are defined, you can easily intersect them with geometric primitives
\\

Example of sphere ray intersection equation.
\myfigure{figures/banana.png}{4in}




\section{Previous Work}

	\subsection{Infinitesimally Slow Rotations}

	\subsection{Inelastic Collisions}

	\subsection{Kinetic Energy Limitation}

\section{Adaption to Finite Angular Velocities}

	\subsection{Parametric Equation Modification}

		\subsubsection{Free Fall Equation}

		\subsubsection{Rolling Equation}

		\subsubsection{Concurrent Rotation \& Rolling Equation}

		\subsubsection{Assumption \#1 - No concurrent Rotation + Freefall}

	\subsection{Elastic Collisions}

		\subsubsection{Planar Collision}

		\subsubsection{Rolling-Edge Collision}

		\subsubsection{Rolling-Corner Collision}

	\subsection{Conservation of Momentum}

		\subsubsection{Settling Guarantee}

		\subsubsection{Duration of Simulation}

		%\subsubsubsection{Concave Vertex}

		%\subsubsubsection{Workpiece Exit}

\section{Results}

	\subsection{Run Time}

	\subsection{Accuracy Comparison}

		\subsubsection{With Euler Integration}

\section{Future Work \& Discussion}

	\subsection{Bounding Box Method Adaption}

	\subsection{Bounded Simulation Between Limits}


